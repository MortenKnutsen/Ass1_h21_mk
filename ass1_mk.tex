% Options for packages loaded elsewhere
\PassOptionsToPackage{unicode}{hyperref}
\PassOptionsToPackage{hyphens}{url}
%
\documentclass[
  12pt,
  norsk,
]{article}
\usepackage{amsmath,amssymb}
\usepackage{lmodern}
\usepackage{ifxetex,ifluatex}
\ifnum 0\ifxetex 1\fi\ifluatex 1\fi=0 % if pdftex
  \usepackage[T1]{fontenc}
  \usepackage[utf8]{inputenc}
  \usepackage{textcomp} % provide euro and other symbols
\else % if luatex or xetex
  \usepackage{unicode-math}
  \defaultfontfeatures{Scale=MatchLowercase}
  \defaultfontfeatures[\rmfamily]{Ligatures=TeX,Scale=1}
\fi
% Use upquote if available, for straight quotes in verbatim environments
\IfFileExists{upquote.sty}{\usepackage{upquote}}{}
\IfFileExists{microtype.sty}{% use microtype if available
  \usepackage[]{microtype}
  \UseMicrotypeSet[protrusion]{basicmath} % disable protrusion for tt fonts
}{}
\makeatletter
\@ifundefined{KOMAClassName}{% if non-KOMA class
  \IfFileExists{parskip.sty}{%
    \usepackage{parskip}
  }{% else
    \setlength{\parindent}{0pt}
    \setlength{\parskip}{6pt plus 2pt minus 1pt}}
}{% if KOMA class
  \KOMAoptions{parskip=half}}
\makeatother
\usepackage{xcolor}
\IfFileExists{xurl.sty}{\usepackage{xurl}}{} % add URL line breaks if available
\IfFileExists{bookmark.sty}{\usepackage{bookmark}}{\usepackage{hyperref}}
\hypersetup{
  pdftitle={«R Notebooks» og reproduserbarhet},
  pdfauthor={Assignment 1 i kurset Data Science 2021 - Karoline Midtbø og Morten Knutsen},
  pdflang={nb-NO},
  hidelinks,
  pdfcreator={LaTeX via pandoc}}
\urlstyle{same} % disable monospaced font for URLs
\usepackage[margin=1in]{geometry}
\usepackage{graphicx}
\makeatletter
\def\maxwidth{\ifdim\Gin@nat@width>\linewidth\linewidth\else\Gin@nat@width\fi}
\def\maxheight{\ifdim\Gin@nat@height>\textheight\textheight\else\Gin@nat@height\fi}
\makeatother
% Scale images if necessary, so that they will not overflow the page
% margins by default, and it is still possible to overwrite the defaults
% using explicit options in \includegraphics[width, height, ...]{}
\setkeys{Gin}{width=\maxwidth,height=\maxheight,keepaspectratio}
% Set default figure placement to htbp
\makeatletter
\def\fps@figure{htbp}
\makeatother
\setlength{\emergencystretch}{3em} % prevent overfull lines
\providecommand{\tightlist}{%
  \setlength{\itemsep}{0pt}\setlength{\parskip}{0pt}}
\setcounter{secnumdepth}{-\maxdimen} % remove section numbering
\ifxetex
  % Load polyglossia as late as possible: uses bidi with RTL langages (e.g. Hebrew, Arabic)
  \usepackage{polyglossia}
  \setmainlanguage[]{norsk}
\else
  \usepackage[main=norsk]{babel}
% get rid of language-specific shorthands (see #6817):
\let\LanguageShortHands\languageshorthands
\def\languageshorthands#1{}
\fi
\ifluatex
  \usepackage{selnolig}  % disable illegal ligatures
\fi
\newlength{\cslhangindent}
\setlength{\cslhangindent}{1.5em}
\newlength{\csllabelwidth}
\setlength{\csllabelwidth}{3em}
\newenvironment{CSLReferences}[2] % #1 hanging-ident, #2 entry spacing
 {% don't indent paragraphs
  \setlength{\parindent}{0pt}
  % turn on hanging indent if param 1 is 1
  \ifodd #1 \everypar{\setlength{\hangindent}{\cslhangindent}}\ignorespaces\fi
  % set entry spacing
  \ifnum #2 > 0
  \setlength{\parskip}{#2\baselineskip}
  \fi
 }%
 {}
\usepackage{calc}
\newcommand{\CSLBlock}[1]{#1\hfill\break}
\newcommand{\CSLLeftMargin}[1]{\parbox[t]{\csllabelwidth}{#1}}
\newcommand{\CSLRightInline}[1]{\parbox[t]{\linewidth - \csllabelwidth}{#1}\break}
\newcommand{\CSLIndent}[1]{\hspace{\cslhangindent}#1}

\title{«R Notebooks» og reproduserbarhet}
\author{Assignment 1 i kurset Data Science 2021 - Karoline Midtbø og
Morten Knutsen}
\date{}

\begin{document}
\maketitle

\hypertarget{introduction}{%
\section{Introduction}\label{introduction}}

In this paper we will look at reproducibility and how R notebook can be
a solution to this problem. First we are going to look at literature
review, where we present what other scientific authors writes about
reproducing. Then we will have our on discussion on how the necessity of
reproducibility in research and whether the use of ``R - Notebooks'' is
a possible solution to the problem of lack of reproducibility. In the
end we will represent a conclusion to the chapter discussion.

When we are talking about \textbf{\emph{reproducibility}} is about
getting confidence in the conclusion to the scientists
(\protect\hyperlink{ref-mcnutt2014}{McNutt, 2014}). The definition of
reproducibility is how other researchers can use the analysis of former
researchers to achieve the same findings using the same analysis and
data
(\protect\hyperlink{ref-samotadavey2021}{\textbf{samotadavey2021?}})

\textbf{\emph{R Notebook}} is a document from R Markdown that contains
chunks (\protect\hyperlink{ref-grolemund2021}{\textbf{grolemund2021?}}).
R Notebook is a document that has direct interaction with R, but it is
also a document that are reproducible
(\protect\hyperlink{ref-Grolemund2021}{\textbf{Grolemund2021?}}). When
you are going to publish the document you can publish Immediately or you
can knitted to another document like HTML, PDF or Word.

\hypertarget{short-literature-review}{%
\section{Short literature review}\label{short-literature-review}}

Roger D. Peng has written a paper that tells us more about
reproducibility. He says that ``\emph{A critical barrier to
reproducibility in many cases is that the computer code is no longer
available}'' (Peng, 2011). This is one of the problems to
reproducibility. There are many researchers that is calling for
reproducibility to be an attainable minimum standard evaluating the
value of scientific claims. Peng also says that ``\emph{Researchers
across a range of computational science disciplines have been calling
for reproducibility, or reproducible research, as an attainable minimum
standard for assessing the value of scientific claims}'' (Peng, 2011).
Even if reproducibility becomes a minimum standard, it does not
guarantee the quality. The ``R'' kite-mark is to indicate the idea that
a knowledgeable has reviewed the data and code and found it reproducible
(Peng, 2011). To make researches reproducible it is recommended for
everyone that use any computing in there research to publish there code.
Even though the code isn't clean, they should publish it, it just need
to be available (Peng 2011). \protect\hyperlink{ref-peng2011}{Peng}
(\protect\hyperlink{ref-peng2011}{2011})

There is on article about ``\emph{Do economics journal archives promote
replicable research?}'' written by McCullough et al.
\protect\hyperlink{ref-mccullough2008}{McCullough et al.}
(\protect\hyperlink{ref-mccullough2008}{2008}). The article show how the
data and the code to an article is important to have to replicate. There
is several companies that are publishing articles that want a system
that makes the authors to include the data and code when they publish,
but most of them fail to achieve that. There are a replication policies
that they can follow, that includes some requirements, on how to do it,
and then have them in archives. The reason to many of them failed is
that i was all up to the authors to do them right. The goal of
replication is that authors or researchers can use other-minded articles
to explore further, so they can avoid wasting time doing the same
research. They also talks how the economics don't see the reason to
replicate, but what else is the meaning of archives?

Code chunks are a series of commands in different programming language,
in example R. Code chunks preform calculations needed to produce the
appropriate output. Also to create intermediate results used across
different code chunks.

A text chunk, on the other hand, describes the results, codes, problems
and the interpretation. Text chunks is formatted for the user to read
it, not the computer.

\hypertarget{discussion}{%
\section{Discussion}\label{discussion}}

R notebook can be a solution to fix the problem of reproducibility, but
only partly. R notebook has the opportunity to be a good tool for any
researcher, but it requires that the researchers knows how to use the
program and they need to have the exact same packages that was used
under the study, or else they will not manage to reproduce the study.
When you are using R studio you can use Github to store your repository,
then you can store it as private or public. After you store it in Github
you can pull it down to R studio whenever you need it. Other Github
users can also use your repository if you make it public, and use your
data and code, if they have the same packages you used. There is also a
lot researchers that want to protect their work and will not make their
codes available, and then it will be difficult to reproduce for others.
For beginners in R there will be a lot to get into, there is many
different things you need to know before you can use it, there is a lot
of programs you need to install for making it optimal.

\begin{enumerate}
\def\labelenumi{\arabic{enumi}.}
\tightlist
\item
  R Notebook will solve the problem with reproducibility
\end{enumerate}

\begin{itemize}
\tightlist
\item
  With R Notebook the document already contain codes.
\item
  It is a free program to use, everybody can install it.
\item
  You can use Github to store it longer.
\end{itemize}

\begin{enumerate}
\def\labelenumi{\arabic{enumi}.}
\setcounter{enumi}{1}
\tightlist
\item
  R notebook will not solve the problem reproducibility
\end{enumerate}

\begin{itemize}
\tightlist
\item
  If you want the data and code, you have to have the right packages
  that was used.\\
\item
  To use the program you have to know how to use it.
\end{itemize}

\texttt{\{r-første\ chunk\}\ sessionInfo()}

This function can help us to reproduce the research because it gives us
information about which R version and packages we used. It can take time
to install all the packages if you don't have them, but with chunks it
will be easier to find which packages we used in our work.

After the literature work we did, we find a lot of information about how
researchers feel about reproducibility. They want to protect the work
they did, and often they don't want to publish their code and data they
used to find their answers. Without the code and data other researchers
can't get the exact same answer, and they will have to take the same
exact test and use a lot of time to get the information. If the author
publish the code and data it will be easier for the next researcher to
just use the test that is already done and they will have the exact same
answer and can use it to develop it. When authors or researchers publish
with data and code, their will be a lot more of opportunities of
reproducibility.

\hypertarget{conclusion}{%
\section{Conclusion}\label{conclusion}}

\hypertarget{references}{%
\section*{References}\label{references}}
\addcontentsline{toc}{section}{References}

\hypertarget{refs}{}
\begin{CSLReferences}{1}{0}
\leavevmode\hypertarget{ref-mccullough2008}{}%
McCullough, B. D., McGeary, K. A., og Harrison, T. D. (2008). Do
Economics Journal Archives Promote Replicable Research? \emph{Canadian
Journal of Economics/Revue canadienne d'économique}, \emph{41}(4),
1406--1420. \url{https://doi.org/10.1111/j.1540-5982.2008.00509.x}

\leavevmode\hypertarget{ref-mcnutt2014}{}%
McNutt, M. (2014). Reproducibility. \emph{Science}, \emph{343}(6168),
229--229. \url{https://doi.org/10.1126/science.1250475}

\leavevmode\hypertarget{ref-peng2011}{}%
Peng, R. D. (2011). Reproducible {Research} in {Computational Science}.
\emph{Science}, \emph{334}(6060), 1226--1227.
\url{https://doi.org/10.1126/science.1213847}

\end{CSLReferences}

\end{document}
